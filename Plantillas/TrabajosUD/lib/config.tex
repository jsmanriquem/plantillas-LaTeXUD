\graphicspath{{fig/}}
\addbibresource{biblio.bib}
\usepgfplotslibrary{colorbrewer}
\pgfplotsset{width=8cm,compat=1.9}
\definecolor{negro}{RGB}{0,0,0}
\definecolor{azul}{RGB}{0,0,255}
\definecolor{verde}{RGB}{0,255,0}
\definecolor{cian}{RGB}{0,255,255}
\definecolor{rojo}{RGB}{255,0,0}
\definecolor{magenta}{RGB}{255,0,255}
\definecolor{amarillo}{RGB}{255,255,0}
\definecolor{blanco}{RGB}{255,255,255}

\definecolor{limegreen}{cmyk}{0.75,0,1,0}
\definecolor{orange}{cmyk}{0,0.8,0.95,0}
\definecolor{purple}{cmyk}{0.75,1,0,0}

\newcommand{\uvec}[1]{~ \boldsymbol{\hat{\textbf{#1}}}}

\newsavebox\foobox
\newlength{\foodim}
\newcommand{\slantbox}[2][0]{\mbox{%
        \sbox{\foobox}{#2}%
        \foodim=#1\wd\foobox
        \hskip \wd\foobox
        \hskip -0.5\foodim
        \pdfsave
        \pdfsetmatrix{1 0 #1 1}%
        \llap{\usebox{\foobox}}%
        \pdfrestore
        \hskip 0.5\foodim
}}
\def\Laplacesymb{\slantbox[-.45]{$\mathscr{L}$}}
\def\inverseLaplacesymb{\Laplacesymb^{-1}}
\def\Fouriersymb{\slantbox[-.45]{$\mathscr{F}$}}
\def\Fouriersinsymb{\Fouriersymb_{\textup{s}}}
\def\Fouriercossymb{\Fouriersymb_{\textup{c}}}
\def\inverseFouriersymb{\Fouriersymb^{-1}}

\newcommand{\Laplace}[2]{
    \IfEqCase{#1}{
        {}{\Laplacesymb\left\lbrace#2\right\rbrace}
        {i}{\inverseLaplacesymb\left\lbrace#2\right\rbrace}
    }[\PackageError{tree}{Undefined option to tree: #1}{}]
}
\newcommand{\Fourier}[2]{
    \IfEqCase{#1}{
        {}{\Fouriersymb\left\lbrace#2\right\rbrace}
        {i}{\inverseFouriersymb\left\lbrace#2\right\rbrace}
        {sin}{\Fouriersinsymb\left\lbrace#2\right\rbrace}
        {cos}{\Fouriercossymb\left\lbrace#2\right\rbrace}
    }[\PackageError{tree}{Undefined option to tree: #1}{}]
}

\sisetup{
  detect-family,
  per-mode = symbol,
  separate-uncertainty = true,
  inter-unit-product = \cdot,
  %output-decimal-marker = {,},
  %exponent-product = \cdot
}

\newcommand{\contenidor}{}
\newcommand{\contenidoa}{}

\newcommand{\resumencommand}{
    \ifdefempty{\contenidor}{}{\begin{center} \Large \textbf{Resumen} \end{center} \vspace*{2pt}

    \hspace{12pt}\begin{minipage}[H]{0.95\textwidth}

    \contenidor

    \end{minipage}
    
    \hspace{11.5pt}\textbf{Palabras clave:} <<Texto>>}

    \ifdefempty{\contenidoa}{\centerline{\rule{0.95\textwidth}{0.4pt}}
    \vspace{15pt}}{\begin{center} \Large \textbf{Abstract} \end{center}
            
    \hspace{12pt}\begin{minipage}[H]{0.95\textwidth}
    
    \contenidoa
    
    \end{minipage}
    
    \hspace{11.5pt}\textbf{Keywords:} <<Text>>
    
    \centerline{\rule{0.95\textwidth}{0.4pt}}
    \vspace{15pt}}
}

\NewEnviron{resumen}{
    \global\renewcommand{\contenidor}{\BODY}
    \resumencommand
}

\NewEnviron{abs}{
    \global\renewcommand{\contenidoa}{\BODY}
    \resumencommand
}

\newcommand{\ddd}[1]{
    \ifx\@empty#1\relax
        % Si el argumento está vacío, no se imprime nada
    \else
        \begin{center} \Large \textbf{Resumen} \end{center}
        \hspace{12pt}\begin{minipage}[H]{0.95\textwidth}
        #1
        \end{minipage}
        
        \hspace{11.5pt}\textbf{Palabras clave:} <<Texto>>
    \fi
}

