\documentclass[letterpaper,12pt]{article}
\usepackage[left=25mm, right=25mm, top=35mm, bottom=30mm, headheight=60pt, includehead]{geometry}
\usepackage{array}
\usepackage{booktabs}
\usepackage{fancyhdr}
\usepackage{graphicx}
\usepackage{calc}
\usepackage{lipsum}
\usepackage{tabularx}
\usepackage{etoolbox}

\newcommand{\respuesta}[1]{
  \ifstrempty{#1}
    {La respuesta es incorrecta.}
    {La respuesta es correcta.}
}

\newlength{\tableheight}

\begin{document}

\settoheight{\tableheight}{\begin{tabularx}{\textwidth}{| c | X | X | X |}
    \hline
    \textbf{Nombre del espacio académico} & \multicolumn{3}{l}{hola} \\ \hline
    \textbf{Código del espacio} & hola & \textbf{Número de créditos} & hola \\ \hline
    \textbf{Código del espacio} & hola & \textbf{Número de créditos} & hola \\ \hline
    \textbf{Código del espacio} & hola & \textbf{Número de créditos} & hola \\ \hline
    \textbf{Código del espacio} & hola & \textbf{Número de créditos} & hola \\ \hline
    \textbf{Código del espacio} & hola & \textbf{Número de créditos} & hola \\ \hline
    \textbf{Código del espacio} & hola & \textbf{Número de créditos} & hola \\ \hline
    \textbf{Código del espacio} & hola & \textbf{Número de créditos} & hola \\ \hline
\end{tabularx}}

La altura de la tabla es de \the\tableheight\ puntos.

\begin{tabularx}{\textwidth}{|X|X|X|}
    \hline
    \multicolumn{2}{|c|}{Celda combinada} & Otra celda \\
    \hline
    Contenido & Contenido & Contenido \\
    \hline
\end{tabularx}

\respuesta{Sí} % muestra "La respuesta es correcta."
\respuesta{} % muestra "La respuesta es incorrecta."

\end{document}

