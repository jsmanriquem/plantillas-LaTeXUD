%%%%%%%%%%%%%%%%%%%%%%%%%%%%%%%%%%%%%%%%%%%%%%%%%%%%%%%%%%%%%%%%%%%%%%%%%%%%
%               PROPUESTA DE TRABAJO DE GRADO (PLANTILLA)                  %
%                                                                          %
%   Hola, esta es una plantilla diseñada para escribir tú propuesta de     %
%   trabajo de grado. Esta plantilla no requiere conocimientos avanzados   %
%   en LaTeX más allá de lo que quieras implementar en tú propuesta.       %
%   El uso del logo de la universidad junto a los pie de página están      %
%   puestos con base en los membretes oficiales dispuestos en              %
%   https://editorial.udistrital.edu.co/mii_desmembrete.php.               %
%                                                                          %
%   Esta plantilla fue diseñada en su totalidad por Sebastian Manrique,    %
%   estudiante de Física de la Universidad Distrital Francisco José de     %
%   Caldas.                                                                %
%                                                                          %
%%%%%%%%%%%%%%%%%%%%%%%%%%%%%%%%%%%%%%%%%%%%%%%%%%%%%%%%%%%%%%%%%%%%%%%%%%%%

% ==========================================================================
% DEFINIR PROGRAMA Y MODALIDAD
% ==========================================================================
% En este apartado es necesario que especifiques el programa y la modalidad
% bajo las siguientes convenciones:
% - Para el programa:
%   * biologia: Programa Académico de Biología.
%   * fisica: Programa Académico de Física.
%   * matematicas: Programa Académico de Matemáticas.
%   * quimica: Programa Académico de Química.
%   Es decir, el nombre de cada programa en minúscula y sin tildes.
% - Para la modalidad:
%   * investigacion: Investigación, Investigación-Creación, Innovación.
%   * monografia: Monografía.
%   * articulo: Producción de Artículos Académicos.
%   * emprendimiento: Proyecto de emprendimiento.
%   * pasantia: Pasantía.
% Se indican dentro del corchete cuadrado de la forma [programa,modalidad].

\documentclass[fisica,monografia]{src/plantilla-tg}

% ==========================================================================
% DATOS BÁSICOS DE AUTORES Y DIRECTORES
% ==========================================================================

\begin{document}

% Acá debes llenar la fecha, esta debe estar en minúscula y sin formato.

\begin{fecha}
    \day{26} % Día
    \month{agosto} % Mes
    \year{2024} % Año
\end{fecha}

% A partir de acá llenas los datos básicos tuyos y de tus directores:
% Nota: para el primer corchete o llaves del apartado \identificacion se
% debe llenar a partir de la siguiente convención:
% - ti: T.I. (tarjeta de identidad).
% - ce: C.E. (cédula de extranjería).
% - <vacío>: C.C. (cédula de ciudadanía).

\begin{autor1}
    % El nombre del primer autor
    \nombre{Sebastian Manrique}
    % La identificación del primer autor
    \identificacion{}{123456}
    % El código estudiantil del primer autor
    \codigo{123456789}
    % El correo electrónico institucional del primer autor
    \correo{jsmanriquem@udistrital.edu.co}
    % El programa al que pertenece el primer autor
    \programa{Física}
\end{autor1}

% Si tú trabajo de grado consta de un solo autor, comenta o elimina todo
% el entorno de "autor2".

\begin{autor2}
    % El nombre del segundo autor
    \nombre{Sebastian Manrique}
    % La identificación del segundo autor
    \identificacion{}{123456}
    % El código estudiantil del segundo autor
    \codigo{123456789}
    % El correo electrónico institucional del segundo autor
    \correo{jsmanriquem@udistrital.edu.co}
    % El programa al que pertenece el segundo autor
    \programa{Física}
\end{autor2}

\begin{director}
    % El nombre del director
    \nombre{Juan Moreno}
    % La identificación del director
    \identificacion{}{123456789}
    % El correo electrónico institucional del director
    \correo{director@universidad.edu.co}
    % La vinculación del director
    \vinculacion{Su universidad o facultad}
\end{director}

% Si tú trabajo de grado consta de un solo director, comenta o elimina
% todo el entorno de "codirector".

\begin{codirector}
    % El nombre del codirector
    \nombre{Juan Moreno}
    % La identificación del codirector
    \identificacion{}{123456789}
    % El correo electrónico institucional del codirector
    \correo{codirector@universidad.edu.co}
    % La vinculación del codirector
    \vinculacion{Su universidad o facultad}
\end{codirector}

%%%%%%%%%%%%%%%%%%%%%%%%%%%%%%%%%%%%%%%%%%%%%%%%%%%%%%%%%%%%%%%%%%%%%%%%%%%%
%                                                                          %
%   Acá termina el llenado de datos básicos, lo siguiente son dos          %
%   configuraciones especiales que debes poner para ciertas modalidades    %
%   de trabajo de grado, lee muy bien si es tú caso, si no es así, este    %
%   archivo termina aquí para ti, de momento.                              %
%                                                                          %
%   Dirígete ahora a la carpeta llamada 'modalidad' ahí selecciona el      %
%   archivo '.tex' que requieres con base en la modalidad que elegiste,    %
%   su nombre es el mismo que tiene la convención que usaste al inicio     %
%   para determinar la modalidad. A partir de ahí, llena cada entorno      %
%   con el contenido de tú propuesta de trabajo de grado.                  %
%                                                                          %
%%%%%%%%%%%%%%%%%%%%%%%%%%%%%%%%%%%%%%%%%%%%%%%%%%%%%%%%%%%%%%%%%%%%%%%%%%%%


% ==========================================================================
% PARA LA MODALIDAD DE INVESTIGACIÓN, INVESTIGACIÓN-CREACIÓN, INNOVACIÓN
% ==========================================================================

% En caso de que NO tenga financiamiento su proyecto, comente la línea
% "\presupuesto" caso contrario, no comente la línea

\presupuesto

% ==========================================================================
% PARA LA MODALIDAD DE PRODUCCIÓN DE ARTÍCULOS ACADÉMICOS
% ==========================================================================

% A continuación, indique el tipo de artículo con base en las siguientes
% convenciones:
% - ciencia: Artículo de investigación científica o tecnológica
% - revision: Artículo de reflexión o de revisión

\tipo{ciencia}

% ==========================================================================
\print % ¡PROHIBIDO ELIMINAR!
% ==========================================================================

\end{document}