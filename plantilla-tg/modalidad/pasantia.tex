\titulo{Título de la pasantía}

\begin{problema}
    \lipsum[1-5]
\end{problema}

\begin{justificacion}
    \lipsum[1-5]
\end{justificacion}

\begin{objetivo_general}
    Un objetivo general
\end{objetivo_general}

\begin{objetivos_especificos}
    \item Un objetivo específico
    \item Dos objetivos específicos
    \item Tres objetivos específicos
\end{objetivos_especificos}

\begin{plan_de_trabajo}
    \lipsum[1-5]
\end{plan_de_trabajo}

\begin{resultados_esperados}
    \lipsum[1-5]
\end{resultados_esperados}

\begin{cronograma}
    \actividad{1}{5}{Activdad uno}
    \actividad{6}{12}{Actividad dos}
    \actividad{13}{25}{Actividad tres}
    \actividad{26}{26}{
        \textbf{Actividad} \textit{con} \texttt{texto} formateado.
        \begin{itemize}
            \item También admite listas internamente.
            \item Finalmente, si la actividad es de una sola semana, al poner el mismo número dos veces lo detecta y ajusta en el texto.
        \end{itemize}
    }
\end{cronograma}